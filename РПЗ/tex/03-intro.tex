\section*{ВВЕДЕНИЕ}
\addcontentsline{toc}{section}{ВВЕДЕНИЕ}

iOS \cite{ios} --- мобильная операционная система, разработанная и выпущенная компанией Apple \cite{apple} в 2007 году: инновационными в первых iPhone \cite{iphone} стали возможность просмотра видео на YouTube \cite{youtube} и поиск нескольких интернет--игр. За 15 лет система претерпела массу изменений: были разработаны новые технологии, создан язык программирования Swift \cite{swift}, который пришел на смену первоначально использовавшемуся Objective--C \cite{objc}, были выпущены гайдлайны \cite{hig}, и всё это --- чтобы вырасти в многофункциональную платформу и превратить iPhone в некий <<ПК в кармане>>. Это позволило Apple заиметь большое количество пользователей по всему миру, а, как следствие, и IT--специалистов, желающих создавать мобильные продукты под iOS. 

Неотъемлемой частью iOS--разработки является создание интерфейса мобильного приложения. Каждый iOS--разработчик сталкивается с проблемой выбора метода разметки экранов продукта: Apple предоставляет несколько вариантов верстки. Возникает вопрос: по каким критериям осуществлять выбор и какая технология является в конкретном случае наиболее подходящей?

Целью данной работы является классификация методов верстки элементов интерфейса в разработке мобильных приложений под iOS. Для достижения поставленной цели необходимо решить следующие задачи: 

\begin{itemize}[label=---]
	\item определить критерии, по которым осуществляется анализ эффективности использования методов верстки элементов интерфейса;
	\item провести обзор существующих методов верстки мобильных приложений под iOS и выделить наиболее популярные;
	\item классифицировать методы на основе выделенных критериев.
\end{itemize}
\pagebreak