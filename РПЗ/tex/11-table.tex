\chapter{Результат анализа}

Результаты анализа представлены в таблицах \ref{result} -- \ref{result2}.

\begin{table}[H]
	\centering
	\caption{Результат анализа}
	\label{result}
	\begin{tabular}{|p{3.3cm}|p{2.7cm}|p{9.2cm}|}
		\hline
		\textbf{Признак классификации} & \textbf{Название метода} & \textbf{Характеристика метода} \\
		\hline
		\multirow{2}{2.9cm}{Ручная верстка} & 
		Frame & 
		Метод требует отдельной обработки каждой вариации размеров экрана в связи с заданием константными значениями размеров и 
		координат элементов интерфейса, однако этот же фактор предоставляет возможность разделения задачи верстки на подзадачи (в 
		пределах одного представления) и упрощается процесс внесения изменений в ранее написанный код. 
		Поскольку frame предполагает программную верстку, разработчику открывается возможность оперировать любыми параметрами UI-элемента. 
		Скорость работы метода совпадает со скоростью размещения элементов на экране, так как не требует дополнительных вычислений.\\
		\hline
		
		\multirow{2}{2.9cm}{Автомати-\linebreakческая верстка} & 
		Autolayout & 
		При грамотно составленных ограничениях, посредством которых элементы располагаются на экране, метод предоставляет 
		возможность создавать интерфейс, масштабируемый для всех типов устройств. 
		Командная разработка возможна в пределах одного родительского представления, внесение изменений в размеры или координаты одного элемента может повлечь за собой внесение 
		изменений и в другие элементы сцены. 
		Поскольку Autolayout предполагает программную верстку, разработчику открывается возможность оперировать любыми параметрами UI-элемента. 
		Время работы метода увеличивается при увеличении взаимосвязей между объектами сцены.\\
		\hline
	\end{tabular}
\end{table}

\begin{table}[H]
	\centering
	\caption{Результат анализа -- продолжение}
	\label{result2}
	\begin{tabular}{|p{3.3cm}|p{2.7cm}|p{9.2cm}|}
		\hline
		\textbf{Признак классификации} & \textbf{Название метода} & \textbf{Характеристика метода} \\
		\hline
		\multirow{2}{2.9cm}{Автомати-\linebreakческая верстка} & Interface Builder & 
		При грамотно составленных ограничениях, посредством которых элементы располагаются на экране, метод предоставляет 
		возможность создавать интерфейс, масштабируемый для всех типов устройств. 
		Командная разработка возможна в пределах одного экрана. Верстка предполагает работу с инструментом создания интерфейсов, а не написание кода, поэтому явно проследить 
		правила задания ограничений возможности нет, что усложняет задачу внесения изменений в размеры или координаты, 
		а также редактирование одного элемента может повлечь за собой внесение правок в другие. 
		Interface Builder предоставляет ограниченный набор параметров графического элемента интерфейса, доступных для обработки. 
		Время работы метода увеличивается при увеличении взаимосвязей между объектами сцены, а также зависит от времени преобразования графического элемента в код.\\
		\hline
	\end{tabular}
\end{table}

Классификация методов верстки интерфейса мобильного приложения приведена в таблице \ref{result3}.



%\begin{sidewaysfigure}
\begin{landscape}
%\begin{pdflscape}

\begin{table}[H]
	\centering
	\caption{Классификация методов верстки интерфейса мобильного приложения}
	\label{result3}
		\begin{tabular}{|p{3.3cm}|p{3.3cm}|p{3.5cm}|p{3.5cm}|p{3.5cm}|p{3.5cm}|p{3.2cm}|}
			\hline
			\textbf{Класс \linebreakметода} & \textbf{Название метода} & \textbf{Масштаби-\linebreakруемость  \linebreakинтерфейса} & 
			\textbf{Возможность \linebreakкомандной \linebreakразработки} & \textbf{Сложность внесения \linebreakизменений} & 
			\textbf{Возможность обработки параметров \linebreak UI-элемента} & \textbf{Скорость \linebreakработы  \linebreakметода} \\
			\hline
			{Метод ручной верстки} & 
			Frame & 
			Нет & Да & Низкая & Да & Высокая \\
			\hline
			\multirow{2}{2.9cm}{Метод \linebreakавтомати-\linebreakческой \linebreakверстки} 
			& Auto-\linebreak layout 
			& Да & Да & Высокая & Да & Средняя \\
			\cline{2-7} & Interface Builder 
			& Да & Да & Высокая & Нет & Низкая \\
			\hline
		\end{tabular}
\end{table}
%\end{sidewaysfigure}
%\end{pdflscape}
\end{landscape}
