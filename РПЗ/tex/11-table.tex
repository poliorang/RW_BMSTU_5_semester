\chapter{Результат анализа}

\begin{table}[H]
	\centering
	\caption{Методы верстки}
	\label{mko_table_0}
	\begin{tabular}{|p{2.9cm}|p{2.9cm}|p{9.5cm}|}
		\hline
		\textbf{Критерий} & \textbf{Метод} & \textbf{Заключение} \\
		\hline
		\multirow{2}{2.9cm}{Масштаби-\linebreakруемость интерфейса} & 
		Frame & 
		Ручная верстка предполагает задание констант в качестве координат и размеров элемента экрана.
		На устройствах различного размера и при разной ориентации экрана верстка будет различаться, 
		поэтому появляется необходимость обрабатывать каждую вариацию размеров экрана отдельно.\\
		\cline{2-3} & Autolayout & 
		Значения ограничений вычисляются динамически при внутренних или внешних изменениях. 
		При корректном составлении линейных уравнений, задающих ограничения, интерфейс будет масштабируем на все виды устройств. \\
		\cline{2-3} & Interface Builder & 
		Значения ограничений вычисляются динамически при внутренних или внешних изменениях. 
		При корректном составлении линейных уравнений, задающих ограничения, интерфейс будет масштабируем на все виды устройств. \\
		\hline
	\end{tabular}
\end{table}

\newpage

\begin{table}[H]
	\centering
	\caption{Методы верстки -- продолжение}
	\label{mko_table_0}
	\begin{tabular}{|p{2.9cm}|p{2.9cm}|p{9.5cm}|}
		\hline
		\textbf{Критерий} & \textbf{Метод} & \textbf{Заключение} \\
		\hline
		\multirow{2}{2.9cm}{Возможность командной разработки} & Frame & 
		Координаты view будут заданы константными значениями в коде и координаты одного представления не зависят от координат другого, 
		возникает возможность разделить задачу верстки экрана на подзадачи. 
		Командная разработка в таком случае возможна, причем можно выделить столько подзадач, сколько view будет принадлежать родительскому view. \\
		\cline{2-3} & Autolayout & 
		В пределах одного родительского view координаты и размеры дочерних могут находиться в зависимости друг от друга, поскольку размещение элементов на экране 
		осуществляется посредством правил, задаваемых линейными уравнениями. 
		Командная разработка возможна, но не в пределах одного родительского представления.\\
		\cline{2-3} & Interface Builder & 
		UI-элементы прикрепляются к view в процессе работы с инструментом создания интерфейсов. Наглядно увидеть правила задания ограничений не представляется возможным. 
		Идет неделимая работа над созданием интерфейса экрана: отсутствует возможность извне добавить элементы на subview. 
		Командная разработка возможная лишь в пределах одного экрана. \\
		\hline
	\end{tabular}
\end{table}

\begin{table}[H]
	\centering
	\caption{Методы верстки -- продолжение 2}
	\label{mko_table_0}
	\begin{tabular}{|p{2.9cm}|p{2.9cm}|p{9.5cm}|}
		\hline
		\textbf{Критерий} & \textbf{Метод} & \textbf{Заключение} \\
		\hline
		\multirow{2}{2.9cm}{Возможность внесения изменений} & Frame & 
		Координаты view заданы константными значениями в коде и координаты одного представления не зависят от координат другого. 
		Внесение изменений сводится к изменению констант, поэтому не влечет за собой необходимость изменять координаты и размеры других view.\\
		\cline{2-3} & Autolayout & 
		Координаты и размеры view могут находиться в зависимости друг от друга, поскольку размещение элементов на экране 
		осуществляется посредством правил, задаваемых линейными уравнениями. 
		Изменение одного правила может повлечь за собой необходимость внесения изменений во все ограничения экрана, соответственно, сложность внесения изменений
		зависит от сложности сцены.\\
		\cline{2-3} & Interface Builder & 
		UI-элементы прикрепляются к view в процессе работы с инструментом создания интерфейсов. Наглядно увидеть правила задания ограничений не представляется возможным, 
		в соответствии с чем внесение изменений в готовый проект возможно, однако является задачей, сложность которой зависит от объемности сцены: изменение одного правила может 
		повлечь за собой необходимость редактирования всех ограничений экрана.\\
		\hline
	\end{tabular}
\end{table}

\begin{table}[H]
	\centering
	\caption{Методы верстки -- продолжение 3}
	\label{mko_table_0}
	\begin{tabular}{|p{2.9cm}|p{2.9cm}|p{9.5cm}|}
		\hline
		\textbf{Критерий} & \textbf{Метод} & \textbf{Заключение} \\
		\hline
		\multirow{2}{3.1cm}{Возможность обработки параметров UI-элементов} & Frame & 
		Верстка кодом: имеется возможность доступа к обработке всех параметров UI-элемента.\\
		\cline{2-3} & Autolayout & 
		Верстка кодом: имеется возможность доступа к обработке всех параметров UI-элемента.\\
		\cline{2-3} & Interface Builder & 
		Верстка через инструмент создания интерфейса: ограниченный набор параметров обработки UI-элемента.\\
		\hline
		\multirow{2}{3.1cm}{Скорость работы метода} & Frame & 
		Параметрами свойства frame выступают константы, поэтому скорость работы метода будет соответствовать скорости размещения элементов на экране.\\
		\cline{2-3} & Autolayout & 
		Скорость работы напрямую зависит от сложности сцены: чем больше объектов содержит сцена, тем больше линейных уравнений необходимо 
		решить для размещения элементов на экране.\\
		\cline{2-3} & Interface Builder & 
		Чем больше объектов содержит сцена, тем больше линейных уравнений необходимо решить для размещения элементов на экране 
		и тем больше времени требует процессор для обработки графических элементов, расположенных на ней, и преобразования в код 
		-- скорость работы метода зависит от сложности сцены. \\
		\hline
	\end{tabular}
\end{table}