\chapter*{Заключение}
\addcontentsline{toc}{chapter}{Заключение}

В ходе выполнения научно-исследовательской работы были выполнены следующие задачи: 

\begin{itemize}
	\item определено понятие верстки под iOS;
	\item выявлены критерии классификации методов верстки;
	\item выделены наиболее популярные методы верстки;
	\item проведена классификация на основе выделенных критериев.	
\end{itemize}

Были классифицированы такие методы верстки, как компоновка с помощью frame, Auto Layout программно и с помощью Interface Builder. Выявлено, что при желании создания интерфейса, масштабируемого для устройств разного размера, не рационально использовать Interface Builder -- его преимущества быстроты создания и изменения конкретно взятого элемента проявляются при создании статического экрана, содержащего не много UI-элементов. Наиболее гибким для командной разработки и внесения изменений в написанный ранее код является верстка с помощью frame, хоть она и уступает Auto Layout в количестве строк кода: читаемо, но объемно. А также несомненное преимущество перед инструментом создания пользовательского интерфейса имеет программная компоновка в вопросе создания кастомных элементов -- Interface Builder предоставляет ограниченный набор параметров обработки UI. В отношении скорости работы методов можно сказать однозначно: каждый из них зависит от объема информации проектируемого экрана и сложности каждого его элемента. Однако при равенстве этих параметров автоматическая компоновка будет уступать ручной по времени работы, поскольку потребуется еще и время на решение систем уравнений, используемых в Auto Layout для задания координат и размеров элементов. 

Все методы имеют преимущества и недостатки, поэтому программисту полезно иметь в своем арсенале каждый, уметь анализировать плюсы и минусы способа в конкретном случае и уметь выбирать оптимальный или же сочетать несколько средств создания интерфейса. 
